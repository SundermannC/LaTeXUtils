%%% Depends on "misc/commandutils.tex"

%%% Utility methods to deal with research questions as named variables (and not with numbers)
%%% This hopefully helps to avoid ambiguity and confusion, and to allow secure renaming of research question numbers.

%%% internal counter to keep track of how many RQs are defined
\newcounter{rqcounter}
\setcounter{rqcounter}{1}

%%% With \defineRQ you can define your research question by providing
%%% - an identifier you want to use as the first argument
%%% - the question as text as the second argument
%%% The order in which you define your research questions matters as it defines their numbers.
%%% For example:
%%%     \defineRQ{SkyBlue}{Why is the sky blue?}
%%%     \defineRQ{EarthRound}{Why is the earth round?}
%%% will make SkyBlue be your first research question and EarthRound your second.
\newcommand{\defineRQ}[2]
{
	\newcounter{numOfRQ#1}
	\setcounter{numOfRQ#1}{\value{rqcounter}}
	\defineCommandByName{rqValue#1}{#2}
	\refstepcounter{rqcounter}
}

%%% With \refRQ you can reference a research question in your text.
%%% The argument to pass is the identifier of the research question you want to reference.
%%% For example:
%%%     We investigate \refRQ{SkyBlue} and \refRQ{EarthRound}.
%%% yields
%%%     We investigate RQ1 and RQ2.
%%% depending on the order you defined the research questions with \defineRQ.
\newcommand{\refRQ}[1]
{RQ\arabic{numOfRQ#1}}

%%% Gives the total amount of research questions you have.
\newcommand{\refNumRQs}[0]
{\the\numexpr\value{rqcounter}-1\relax}

%%% With \printRQ you can print the research question in your text.
%%% The argument to pass is the identifier of the research question you want to show.
%%% For example:
%%%     Our main questions is: \printRQ{SkyBlue}
%%% yields
%%%     Our main questions is: Why is the sky blue?
\newcommand{\printRQ}[1]
{\runCommandByName{rqValue#1}}

%%% \showRQInDescription is a shortcut to introduce your research questions in a desription environment.
%%% Example usage:
%%% \begin{description}
%%%     \showRQInDescription{SkyBlue}
%%%     \showRQInDescription{EarthRound}
%%% \end{description}
\newcommand{\showRQInDescription}[1]
{\item[\refRQ{#1}] \printRQ{#1}}
